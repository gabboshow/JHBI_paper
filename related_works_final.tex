\section{Related works}

The automatic monitoring and analysis of chronic diseases has always been central in research on wearable sensors. In particular, the continuous monitoring of COPD patients has gained considerable interests in the recent years. Liao et al.~\cite{Liao_2014} provided a review focused on describing current wearable technologies for measuring the physical activity level of COPD patients. Dimensions such as reliability, validity, advantages and limitations are discussed. Of particular interest is the work of Patel et al.~\cite{Patel_2007} where a comparative study of machine learning techniques is presented in order to track changes in physiological responses of COPD patients with respect to their physical activity level. They used motion data to monitor activities in conjunction with heart-rate and respiration rate to capture the physiological responses of the patients while performing a set of activities. Beattie et al.~\cite{Beattie_2014} considered how the early detection of disease exacerbation can lead to earlier provision of intervention advice. These authors focus on important parameters for patients self-management such as autonomy, methods of data transmission and levels of intrusiveness and propose guidelines for the development of a context-aware system aimed at overcoming current limitations in the perspective of a user-friendly system for the patients. More advanced self-management platforms have been recently proposed. For example, Bellos et al.~\cite{Bellos_2014} propose an integrated platform aiming at the effective management and real-time assessment of the health status of COPD patients.
A combination of machine-learning techniques was able to provide real-time categorizations of COPD episodes  and estimate the severity of pathological situations in different levels, triggering an alerting mechanism for the patient and the clinical supervisor. 
%\textbf{Add here the main limitation of these works.Add here the main limitation of these works.Add here the main limitation of these works. Add here the main limitation of these works. Add here the main limitation of these works.}

Topic models represent a class of algorithms able to discover hidden thematic structure in collections of documents. Due to their pattern discovery nature, they have been widely explored in the wearable sensor and activity recognition community. 
Huyn at al.~\cite{Huynh_2008} showed that the activity patterns discovered using topic modelling approaches correspond to high-level users' behavior. Authors used activity patterns based on a learned vocabulary of meaningful events such as walking, using the phone, discussing at whiteboard, etc. These authors also addressed the point of avoiding supervised learning approaches using unsupervised methods for building the vocabulary using a clustering approach. Qualitative results show that high-level structure of the data as well as activity transitions, novelties and anomalies can be discovered using their approach.
Seiter et al.~\cite{Seiter_2014} investigated unsupervised activity discovery approaches using three topic model approaches. Authors analyzed three public datasets with different properties affecting the discovery such as primitive rate, activity composite specificity, primitive sequence similarity, and composite-instance ratio. They compared the activity composite discovery performance against the performance of a k-means clustering algorithms providing guidelines for optimal parameter selection. Results indicated that LDA shows higher robustness against noise compared to k-means and other topic modelling approaches. 
%\textbf{Look for one more supporting the hypothesis of the paper and Add here the motivation why those approaches are not completely suitable for the problem at hand.}