\section{Conclusion}
Unsupervised discovery of latent structures in data from activity sensors is becoming of increased relevance 
due to the increasing amount of available activity data.
The paper contributes to this field in different ways. First of all, real-life data 
is used concerning a relatively large population involving healthy and COPD patients. 
Secondly, the design and usage of tools differs in a number of ways from that reported so far.
Using relatively simple assumptions and settings, it is shown that interpretable and consistent results can be obtained using the 
large set of real-life data. As such it is a encouraging step into the direction of practical applications of these techniques in daily life.
