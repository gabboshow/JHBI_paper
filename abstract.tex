\begin{abstract}
With the growing amount of physical activity (PA) measures available, the need for methods and algorithms to automatically analyze and interpret unannotated data increases. In this paper PA is seen as a combination of multimodal constructs that can co-occur in different way and proportion during the day. The design of a methodology able to integrate and analyze them is discussed and its operation is illustrated by applying it to a data set comprising data from 997 COPD patients and 66 healthy subjects.
The method encompasses different stages. The first stage is a completely automated method of labeling low-level multimodal PA measures. The information contained in the PA labels are further structured using topic modelling techniques, a machine learning method from the text processing community. The topic modelling discovers the main themes that pervade a large set of data, in this case it discovers PA routines that are active in the assessed days of the subjects under study. Applying the designed algorithm to our data provides new learnings and insights. As expected, the algorithm discovers that the PA routines for COPD patients and healthy subjects are substantially different regarding their composition and moments in time in which transitions occur. 
Furthermore, it shows certain consistent trends relating to disease severity as measured by standard clinical practice.


%Inspired by machine learning methods from the text processing community, the daily stream of data coming from the Sensewear device is converted into a series of documents consisting of sets of discrete PA labels. Learning of the labels is performed in a fully unsupervised fashion by mean of clustering techniques. The sets of documents are then mined for common topics, i.e. activity patterns or routines that show the same probability distribution across the labels. Latent Dirichlet Allocation (LDA) is used to both extract the routines from the set of labels and to infer the day of the subjects in order to find the daily routine-activation probabilities that characterize patients and healthy subjects. Results show not only that the routines found are substantially different for COPD patients and healthy subjects, but also that the transitions between routines occur at different time. The discovered activity pattern and differences could be used, from an application point of view, to guide the patients in their rehabilitation program.
\end{abstract} 
