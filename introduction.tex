\section{Introduction}
%02 12 2014

\IEEEPARstart{T}{he} prevalence of chronic diseases in general is rising due to an aging population as well as to  environmental and lifestyle changes.  This is particularly true for respiratory diseases such as Chronic Obstructive Pulmonary Disease (COPD), which is a progressive and irreversible disease that results in airflow limitation and significant extra pulmonary effects which limit physical activities~\cite{Seymour_2009, Verboom_2011}. 
Physical activity (PA), defined as any bodily  movement produced by skeletal muscles that requires energy expenditure~\cite{Caspersen_1985}, is known to be a relevant indicator of COPD patients health state. Research on  physical activity levels of COPD patients has consistently shown that COPD patients have lower physical activity levels than their healthy peers~\cite{Vorrink_2011}. Moreover, reduced levels of PA have been found to be related to an increased risk of hospital admission and mortality among COPD patients~\cite{Garcia_2006, Pitta_2006, Waschki_2011}. Outcome variables related to this type of analysis mainly focused on amount and volume parameters, such as
number of steps, volume of physical activity as expressed by total 
number of counts, and total energy expenditure. Although these are important health markers of patients suffering from COPD, interventions thus far have failed to demonstrate important increases in physical activities in patients with COPD. A better insight into daily PAs of patients with COPD needs to be achieved in order to assist in targeted therapeutic strategies and personalized coaching programs. In order to quantify PA it is not only necessary to measure a set of multidimensional parameters as the one mentioned above, but also their co-occurrence and temporal pattern. Moreover, when analyzing PA, physiological responses such as heart rate, temperature or galvanic skin response should be considered complementary constructs that needs to be harmoniously integrated with PA measures into meaningful descriptors.
With recent improvements in wearable sensor technologies it becomes easier in daily life to acquire massive amounts of different sensors data. At the same time, however, it is difficult to combine and extrapolate meaningful information in absence of any supervision or annotation.
A PA descriptor could be seen as a composite of multiple low-level PA measures, including their physiological responses, that can co-occur in a different way and proportion i.e., PA routines. Routines could also occur at different time and in different proportion across the day for different patients or subgroups of patients characterizing their activity behaviour. The reader might think about PA measures and physiological responses as the letters composing the words that describe PAs. The co-occurrency of these words creates groups of PA constructs describing the main topics that pervade the day of a patient. 
This work aims at studying low-level multimodal sensor data in order to find unknown and characteristic PA structures, able to quantify difference among COPD and healthy subjects (matching for gender age and BMI) and within COPD severity classes. This is particularly difficult in this patient population
since it is known that COPD patients maintain a constant
inactive behaviour during the day that makes standard activity
recognition tools not suitable for the purpose.\\
%In particular, this paper provides the following contributions:
%\begin{enumerate}
%\item We proposes an automatic data mining method based on multimodal physical activity measures able to quantify differences in physical activity routines between 66 COPD patients and 66 healthy control subjects.
%\item We proposes an automatic data mining method based on topic modelling, able to discover activity routines as a probabilitstic combination of multimodal physcal activity measures.
%\item We identify differences in the physical activity routines discovered between 66 COPD patients and 66 healthy control subjects.
%\item We propose LDA as Model of Physical Activity profiles in COPD patients
%\item We identify how activity patterns are distributed across time, what are the typical active periods during daily life of patients living at home in a group of 977 COPD patients and we individuate differences between COPD severity classes.
%\item We propose daily routines as a better identifier of active or sedentary life style. 
%\end{enumerate}
In particular, this paper provides the following contributions: (1) we propose a methodology to create a vocabulary of meaningful words from a set of mutimodal PA measures without the need of any supervision or parameter tuning, (2) we discover PA routines that pervade daily life of COPD patients. Finally, (3) we infer the underlying PA routine structure of numerous patients data quantifying differences between COPD patientsa and healthy subjects and among COPD patients. In particular, for each assessed day we infer which is the distribution over the routines in day segments of 30 minutes, describing in such a way the temporal regularities of the multidimodal PA measures.


%PA should be regarded as a multi-dimensional construct~\cite{Bussmann_2013}, which means that it should be described by relevant constructs and parameters other than the total amount of PA only. 

%Over the last few decades, the methods used to objectively assess
%a person's behavior in terms of body postures (e.g., sitting, standing), body movements(e.g., walking, cycling),and/or daily activities (e.g., sports, gardening) in a daily life setting have improved considerably. Devices have become smaller, power consumption requirements have decreased, data storage capacity has increased, and innovative, integrated sensors have been developed. 

%These developments and outcome variables have contributed to a better understanding of daily behavior and a more accepted role of it in research and clinical practice.




%For comparison the average values of METs and the standard error (SE) during the days for COPD patients and heathy subjects could be observed in Fig.\ref{fig:2}. 
%Previous analysis failed because taking only the average as you can see from Fig.~\ref{fig:2} after 1 hour from the waking up moment the standard error of the mean increases again.



%Moreover, patients are not able to accurately self-report their physical
%activities~\cite{Pitta_2007}. The method, then, needs to be unsupervised without the needs of using annotations.
